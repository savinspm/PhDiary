\usepackage[utf8]{inputenc}
\usepackage{graphicx}
\usepackage{hyperref,cleveref} %Referencias que ponen el tipo de elemento automatico. Es necesario el hyperref antes.
\usepackage{verbatim} % comentarios
\usepackage{float}%Para posicionar las imagenes con el H p.e.

\usepackage[top=25mm,bottom=25mm,left=25mm,right=25mm,headsep=10pt,a4paper]{geometry} % Page margins

\usepackage{titlesec}% Delete "chapter" word in each chapter
\titleformat{\chapter}{\huge}{\thechapter.}{20pt}{\huge}

\usepackage{minitoc} % Little content in each chapter
\usepackage{url}

% DEfiniciones
\usepackage{amsthm}
\theoremstyle{definition}
\newtheorem{definition}{Definition}[section]

% Bibliografia al final de cada capitulo
\usepackage[numbers,square]{natbib}
\bibliographystyle{plain}


%%%%%%%%%%%%%%%%%%
% LISTING
%%%%%%%%%%%%%%%%%%
\usepackage{listings}
\usepackage{color}

\definecolor{codegreen}{rgb}{0,0.6,0}
\definecolor{codegray}{rgb}{0.5,0.5,0.5}
\definecolor{codepurple}{rgb}{0.58,0,0.82}
\definecolor{backcolour}{rgb}{0.95,0.95,0.92}

\lstdefinestyle{mystyle}{
	backgroundcolor=\color{backcolour},   
	breaklines=true,
	commentstyle=\color{codegreen},
	keywordstyle=\color{magenta},
	numberstyle=\tiny\color{codegray},
	stringstyle=\color{codepurple},
	basicstyle=\scriptsize\ttfamily,
	breakatwhitespace=false,         
	breaklines=true,                 
	captionpos=b,                    
	keepspaces=true,                 
	numbers=left,                    
	numbersep=5pt,
	postbreak=\raisebox{0ex}[0ex][0ex]{\ensuremath{\color{red}\hookrightarrow\space}},
	showspaces=false,                
	showstringspaces=false,
	showtabs=false,                  
	tabsize=2
}

\lstset{style=mystyle}
%End Listing

% CABECERA Y PIE DE PAGINA
\usepackage{fancyhdr}

% aqui definimos el encabezado de las paginas pares e impares.
\lhead[\leftmark]{\leftmark}
\chead[]{}
\rhead[]{}
%\renewcommand{\headrulewidth}{0.5pt}

% aqui definimos el pie de pagina de las paginas pares e impares.
\lfoot[]{}
\cfoot[\thepage]{\thepage}
\rfoot[]{}
%\renewcommand{\footrulewidth}{0.5pt}

% aqui definimos el encabezado y pie de pagina de la pagina inicial de un capitulo.
\fancypagestyle{plain}{
	\fancyhead[L]{}
	\fancyhead[C]{}
	\fancyhead[R]{}
	\fancyfoot[L]{}
	\fancyfoot[C]{\thepage}
	\fancyfoot[R]{}
	%\renewcommand{\headrulewidth}{0.5pt}
	%\renewcommand{\footrulewidth}{0.5pt}
}

\pagestyle{fancy} 
% FIN